\documentclass[12pt]{article}
\usepackage{polski}
\usepackage[utf8]{inputenc}
\title{Specyfikacja funkcjonalna symulatora gry w życie}
\author{Krzysztof Piekarczyk, Karol Brzeziński}
\date{03.04.2019}
\begin{document}
\maketitle


\section{Cel projektu}

Program game\_of\_life ma na celu przeprowadzać symulację gry w życie Johna Conwaya, udostępnić użytkownikowi utworzenie początkowej generacji i obserwacje zmieniającej się planszy.

\section{Teoria}

Gra toczy się na planszy o określonej wielkości. Komórki mogą znajdować się w jednym w dwóch stanów: "żywa" lub "martwa". Część komórek na planszy jest "żywa". Gdy martwa komórka ma 3 sąsiadów to staje się żywa w następnej iteracji. Gdy żywa ma 2 lub 3 to jej stan się nie zmienia, a w każdej innej sytuacji umiera.

\section{Możliwości programu}
Początkowa generacja jest podawana przez użytkownika w formie tekstowej do pliku. Część z danych początkowych, takich jak wysokość, szerokość i koordynaty żywych komórek może zostać zastąpiona ścieżką do pliku png przedstawiającego planszę do gry. W przypadku nie podania żadnych żywych komórek zostaną umiejscowione losowo. Program umożliwia przeprowadzenie wybranej przez użytkownika liczby iteracji. W zależności od podanych danych program wygeneruje pliki png lub GIF-a.

\section{Dane wejściowe}

Program potrzebuje danych wejściowych, które przechowywane są w pliku:
\begin{itemize}
	\item width - określa szerokość planszy, musi być dodatnia;
	\item height - określa wysokość planszy, musi być dodatnia;
	\item alive - wybór komórek które mają być żywe;
	\item png - podawane zamiast width, height i alive, wskazuje na plik png z którego skorzysta program aby wygenerować planszę;
	\item freq - używane w przypadku generacji plików png, określa ich częstotliwość tworzenia;
	\item fps - używane w przypadku generacji GIF, określa ilość iteracji na sekundę, maksymalnie może wynieść 60;
	\item max\_iter - ilość przeprowadzonych iteracji symulacji, w przypadku GIF-a maksymalnie 500 iteracji.
\end{itemize}

Wszystkie dane podawane są w sposób: klucz\textgreater{}wartość.
Przykładowy plik wywołania:\\
width\textgreater{}500\\
height\textgreater{}500\\
max\_iter\textgreater{}500\\
freq\textgreater{}50\\
alive\textgreater{}20,30\\
alive\textgreater{}21,30\\
alive\textgreater{}22,30\\
alive\textgreater{}23,30\\
alive\textgreater{}24,30\\

\section{Dane wyjściowe}

\begin{itemize}
	\item pliki png - wytwarzane przy kolejnych iteracjach, zapisują aktualną planszę gry
	\item plik GIF - odtwarzający przebieg gry
\end{itemize}

\section{Wywoływanie programu}

By wywołać program, należy użyć: \textbf{\textit{./gol \textless plik\_konfiguracyjny\textgreater}}.

\section{Opis plików nagłówkowych}
Wszystkie pliki składające się na program znajdują się w dolderze \textbf{\textit{src}}.

    \subsection{\textbf{\textit{game\_space\_structures.h}}}
        Ten plik zawiera typy danych i struktury wykorzystywane przez symulację.

        \begin{itemize}
            \item \textit{enum \textbf{cell\_state}} [\textit{DEAD}=45, \textit{ALIVE}=35] - służy przedstawieniu stanu komórki
            \item \textit{struct \textbf{game\_space\_s} -\textgreater typedef \textbf{game\_space\_t}} - służy przedstawieniu obecnego stanu symulacji jako kontener
            \item \textit{enum \textbf{bool\_e} -\textgreater typedef \textbf{bool\_t}} - implementacja zmiennej boolowskiej
        \end{itemize}


    \subsection{\textbf{\textit{game\_space\_management.h}}}
        Zależności: \textit{ \textbf{game\_space\_structures.h}}\\\\
        Ten plik zawiera funkcje odpowiadające za manipulację stanem symulacji.

        \begin{itemize}
            \item \textit{game\_space\_t *\textbf{create\_blank\_game\_space}} - przyjmuje wymiary planszy, i zwraca gotowy kontener symulacji z pustą planszą o odpowiednich wymiarach
            \item \textit{void \textbf{set\_max\_iterations}} - przyjmuje kontener symulacji i liczbę iteracji, ustawia ilość iteracji które ma prsebyć symulacja
            \item \textit{void \textbf{ramdomise\_game\_space}} - przyjmuje kontener symulacji i randomizuje stan planszy
            \item \textit{void \textbf{plane\_dimension\_guard}} - przyjmuje kontener symulacji i pozycję x i y, sprawdza, czy punkt jest na planszy
            \item \textit{void \textbf{flip\_cell\_state}} - przyjmuje kontener symulacji i pozycję x i y, zmienia stan komórki
            \item \textit{unsigned char \textbf{check\_cell\_state}} - przyjmuje kontener symulacji i pozycję x i y, zwraca stan komórki
            \item \textit{game\_space\_t *\textbf{copy\_game\_space}} - przyjmuje kontener symulacji i zwraca jego dokładną kopie
            \item \textit{void \textbf{free\_game\_space}} - przyjmuje kontener symulacji i zwalnia zajmowaną przez niego pamięć
        \end{itemize}


    \subsection{\textbf{\textit{game\_space\_display.h}}}
        Zależności: \textit{ \textbf{game\_space\_structures.h}}, \textit{ \textbf{gifenc.h}}, \textit{ \textbf{lodepng.h}}\\\\
        Ten plik zawiera funkcje odpowiadające za manipulację stanem symulacji.

        \begin{itemize}
            \item \textit{ge\_GIF *\textbf{create\_gif\_\_timebar}} - przyjmuje kontener symulacji i nazwę dla generowanego pliku, tworzy strukturę reprezentującą gifa przed wyrenderowaniem
            \item \textit{void \textbf{render\_gif\_frame\_\_timebar}} - przyjmuje kontener symulacji i reprezentację gifa, odpowiada za zamianę stanu symulacji na klatkę gifa
            \item \textit{void \textbf{render\_png}} - przyjmuje kontener symulacji i renderuje obraz PNG na podstawie jej akutalnego stanu
        \end{itemize}


    \subsection{\textbf{\textit{game\_functions.h}}}
        Zależności: \textit{ \textbf{game\_space\_structures.h}}, \textit{ \textbf{game\_space\_management.h}}, \textbf{\textit{game\_space\_display.h}}\\\\
        Ten plik zawiera funkcje odpowiadające za symulację.

        \begin{itemize}
            \item \textit{unsigned int *\textbf{count\_live\_neighbors}} - przyjmuje kontener symulacji i koordynaty x i y, zwraca liczbę żywych sąsiadów
            \item \textit{bool\_t \textbf{does\_cell\_die}} - przyjmuje kontener symulacji i koordynaty x i y, określa czy komórka umrze
            \item \textit{bool\_t \textbf{does\_cell\_revive}} - przyjmuje kontener symulacji i koordynaty x i y, określa czy komórka ożyje
            \item \textit{void \textbf{run\_iteration}} - przyjmuje kontener symulacji i wykonuje jedną iterację symulacji
            \item \textit{void \textbf{run\_game\_of\_life\_\_create\_a\_gif\_\_timebar}} - przyjmuje kontener symulacji, nazwę dla pliku wyjściowego oraz liczbę iteracji na sekundę zawartych w gifie, przetwarza symulację i tworzy na jej podstawie plik GIF
            \item \textit{void \textbf{run\_game\_of\_life\_\_create\_pngs}} - przyjmuje kontener symulacji oraz liczbę iteracji co które ma zostać stworzony obraz PNG odwzorowania stanu symulacji, przetwarza symulację i tworzy na jej podstawie pliki PNG
        \end{itemize}


    \subsection{\textbf{\textit{game\_configuration.h}}}
        Zależności: \textit{ \textbf{game\_space\_structures.h}}, \textbf{\textit{game\_functions.h}}, \textbf{\textit{game\_space\_management.h}}\\\\
        Ten plik zawiera funkcje odpowiadające za konfigurację symujacji.

        \begin{itemize}
            \item \textit{game\_space\_t *\textbf{create\_game\_of\_life\_\_png}} - przyjmuję nazwę pliku PNG, i tworzy na jego podstawie kontener symulacji z odtworzonym stanem początkowym planszy (w pełni czarny pixel to jedna żywa komórka)
            \item \textit{void \textbf{run\_game\_of\_life\_\_file}} - przyjmuje nazwę pliku konfiguracyjnego, i na jego podstawie przeprowadza symulację
        \end{itemize}
        
\section{Scenariusz działania programu}

\begin{itemize}
	        \item Uruchomienie programu z podaniem pliku z danymi
	        \item Wczytanie danych z pliku
	        \item Utworzenie planszy gry i wypełnienie jej komórkami
	        \item Przeprowadzenie symulacji gry i zapis do pliku aktualnych stanów planszy
	        \item Zakończenie działania programu
\end{itemize}

\section{Komunikaty błędów}

\begin{itemize}
	\item W przypadku nie podania nazwy pliku z danymi wejściowymi:\\
	      "Lack of a configuration file.
	      Correct usage: path.game\_of\_life.exe config\_file"

	\item Podania nazwy pliku z danymi wejściowymi który 
	nie istnieje:\\
	      "No file 'FileName'."

	\item Braku potrzebnych danych w pliku:\\
	      "Configuration incomplete! Required values:\\
	      - height\\
	      - width\\
	      - png \textless{}instead of height and width\textgreater{}\\
	      - max\_iter\\
	      - freq or fps"
	
	\item Podania niepoprawnych koordynat żywych komórek:\\
	      "Y index too high! (Indexing starts at 0)"
	      
	\item Podania nazwy pliku png który nie istnieje:\\
		  "No file with name: FileName!"
		  
	\item Wystąpi błąd przy zapisie plików png:\\
		  "An error occured while writing file: FileName"

\end{itemize}
Jeżeli wystąpią powyższe błędy program kończy działanie.

\end{document}
