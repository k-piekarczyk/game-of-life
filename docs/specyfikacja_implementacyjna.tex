\documentclass[12pt]{article}
\usepackage{polski}
\usepackage[utf8]{inputenc}
\title{Specyfikacja implementacyjna symulatora gry w życie}
\author{Krzysztof Piekarczyk, Karol Brzeziński}
\date{03.04.2019}
\begin{document}
\maketitle


\section{Informacje ogólne}
By wywołać program, należy użyć: \textbf{\textit{./gol \textless plik\_konfiguracyjny\textgreater}}.
Przykładowe pliki konfiguracyjne znajdują się w folderze \textbf{\textit{demo-conf}}.\\
\\
Przy poprawnym rozpoczęciu, program będzie na bierząco informował o postępach symulacji.


\section{Opis plików nagłówkowych}
Wszystkie pliki składające się na program znajdują się w dolderze \textbf{\textit{src}}.

    \subsection{\textbf{\textit{game\_space\_structures.h}}}
        Ten plik zawiera typy danych i struktury wykorzystywane przez symulację.

        \begin{itemize}
            \item \textit{enum \textbf{cell\_state}} [\textit{DEAD}=45, \textit{ALIVE}=35] - służy przedstawieniu stanu komórki
            \item \textit{struct \textbf{game\_space\_s} -\textgreater typedef \textbf{game\_space\_t}} - służy przedstawieniu obecnego stanu symulacji jako kontener
            \item \textit{enum \textbf{bool\_e} -\textgreater typedef \textbf{bool\_t}} - implementacja zmiennej boolowskiej
        \end{itemize}


    \subsection{\textbf{\textit{game\_space\_management.h}}}
        Zależności: \textit{ \textbf{game\_space\_structures.h}}\\\\
        Ten plik zawiera funkcje odpowiadające za manipulację stanem symulacji.

        \begin{itemize}
            \item \textit{game\_space\_t *\textbf{create\_blank\_game\_space}} - przyjmuje wymiary planszy, i zwraca gotowy kontener symulacji z pustą planszą o odpowiednich wymiarach
            \item \textit{void \textbf{set\_max\_iterations}} - przyjmuje kontener symulacji i liczbę iteracji, ustawia ilość iteracji które ma prsebyć symulacja
            \item \textit{void \textbf{ramdomise\_game\_space}} - przyjmuje kontener symulacji i randomizuje stan planszy
            \item \textit{void \textbf{plane\_dimension\_guard}} - przyjmuje kontener symulacji i pozycję x i y, sprawdza, czy punkt jest na planszy
            \item \textit{void \textbf{flip\_cell\_state}} - przyjmuje kontener symulacji i pozycję x i y, zmienia stan komórki
            \item \textit{unsigned char \textbf{check\_cell\_state}} - przyjmuje kontener symulacji i pozycję x i y, zwraca stan komórki
            \item \textit{game\_space\_t *\textbf{copy\_game\_space}} - przyjmuje kontener symulacji i zwraca jego dokładną kopie
            \item \textit{void \textbf{free\_game\_space}} - przyjmuje kontener symulacji i zwalnia zajmowaną przez niego pamięć
        \end{itemize}


    \subsection{\textbf{\textit{game\_space\_display.h}}}
        Zależności: \textit{ \textbf{game\_space\_structures.h}}, \textit{ \textbf{gifenc.h}}, \textit{ \textbf{lodepng.h}}\\\\
        Ten plik zawiera funkcje odpowiadające za manipulację stanem symulacji.

        \begin{itemize}
            \item \textit{ge\_GIF *\textbf{create\_gif\_\_timebar}} - przyjmuje kontener symulacji i nazwę dla generowanego pliku, tworzy strukturę reprezentującą gifa przed wyrenderowaniem
            \item \textit{void \textbf{render\_gif\_frame\_\_timebar}} - przyjmuje kontener symulacji i reprezentację gifa, odpowiada za zamianę stanu symulacji na klatkę gifa
            \item \textit{void \textbf{render\_png}} - przyjmuje kontener symulacji i renderuje obraz PNG na podstawie jej akutalnego stanu
        \end{itemize}


    \subsection{\textbf{\textit{game\_functions.h}}}
        Zależności: \textit{ \textbf{game\_space\_structures.h}}, \textit{ \textbf{game\_space\_management.h}}, \textbf{\textit{game\_space\_display.h}}\\\\
        Ten plik zawiera funkcje odpowiadające za symulację.

        \begin{itemize}
            \item \textit{unsigned int *\textbf{count\_live\_neighbors}} - przyjmuje kontener symulacji i koordynaty x i y, zwraca liczbę żywych sąsiadów
            \item \textit{bool\_t \textbf{does\_cell\_die}} - przyjmuje kontener symulacji i koordynaty x i y, określa czy komórka umrze
            \item \textit{bool\_t \textbf{does\_cell\_revive}} - przyjmuje kontener symulacji i koordynaty x i y, określa czy komórka ożyje
            \item \textit{void \textbf{run\_iteration}} - przyjmuje kontener symulacji i wykonuje jedną iterację symulacji
            \item \textit{void \textbf{run\_game\_of\_life\_\_create\_a\_gif\_\_timebar}} - przyjmuje kontener symulacji, nazwę dla pliku wyjściowego oraz liczbę iteracji na sekundę zawartych w gifie, przetwarza symulację i tworzy na jej podstawie plik GIF
            \item \textit{void \textbf{run\_game\_of\_life\_\_create\_pngs}} - przyjmuje kontener symulacji oraz liczbę iteracji co które ma zostać stworzony obraz PNG odwzorowania stanu symulacji, przetwarza symulację i tworzy na jej podstawie pliki PNG
        \end{itemize}


    \subsection{\textbf{\textit{game\_configuration.h}}}
        Zależności: \textit{ \textbf{game\_space\_structures.h}}, \textbf{\textit{game\_functions.h}, \textbf{\textit{game\_space\_management.h}\\\\
        Ten plik zawiera funkcje odpowiadające za konfigurację symujacji.

        \begin{itemize}
            \item \textit{game\_space\_t *\textbf{create\_game\_of\_life\_\_png}} - przyjmuję nazwę pliku PNG, i tworzy na jego podstawie kontener symulacji z odtworzonym stanem początkowym planszy (w pełni czarny pixel to jedna żywa komórka)
            \item \textit{void \textbf{run\_game\_of\_life\_\_file}} - przyjmuje nazwę pliku konfiguracyjnego, i na jego podstawie przeprowadza symulację
        \end{itemize}

\end{document}