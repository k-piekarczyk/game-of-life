\documentclass[12pt]{article}
\usepackage{polski}
\usepackage[utf8]{inputenc}
\title{Specyfikacja funkcjonalna symulatora gry w życie}
\author{Krzysztof Piekarczyk, Karol Brzeziński}
\date{03.04.2019}
\begin{document}
\maketitle


\section{Cel projektu}

Program game\_of\_life ma na celu przeprowadzać symulację gry w życie Johna Conwaya. Początkowa generacja jest podawana przez użytkownika lub ustalana losowo. Program powinien umożliwiać przeprowadzenie dowolnej liczby generacji, zapisywać wybrane jako plik png oraz umożliwiać wczytanie wygenerowanych generacji.


\section{Dane wejściowe}

Program potrzebuje danych wejściowych, które przechowywane są w pliku:
\begin{itemize}
	\item width - określa szerokość planszy;
	\item height - określa wysokość planszy;
	\item alive - wybór komórek które mają być żywe;
	\item png - podawane zamiast width, height i alive, wskazuje na plik png z którego skorzysta program;
	\item freq - używane w przypadku generacji png, określa częstotliwość tworzenia png;
	\item fps - używane w przypadku generacji GIF, określa ilość iteracji na sekundę;
	\item max\_iter - ilość przeprowadzonych iteracji symulacji.
\end{itemize}

Wszystkie dane podawane są w sposób: klucz\textgreater{}wartość.
Przykładowy plik wywołania:\\
width\textgreater{}500\\
height\textgreater{}500\\
max\_iter\textgreater{}500\\
freq\textgreater{}50\\
alive\textgreater{}20,30\\
alive\textgreater{}21,30\\
alive\textgreater{}22,30\\
alive\textgreater{}23,30\\
alive\textgreater{}24,30\\
Przykładowe wywołanie programu:	./gol config\_file


\section{Dane wyjściowe}

\begin{itemize}
	\item pliki png - wytwarzane przy kolejnych iteracjach, zapisują aktualną planszę gry
	\item plik GIF - odtwarzający przebieg gry
\end{itemize}


\section{Teoria}

Gra toczy się na planszy o określonej wielkości. Komórki mogą znajdować się w jednym w dwóch stanów: "żywa" lub "martwa". Część komórek na planszy jest "żywa". Gdy martwa komórka ma 3 sąsiadów to staje się żywa w następnej iteracji. Gdy żywa ma 2 lub 3 to jej stan się nie zmienia, a w każdej innej sytuacji umiera.


\section{Komunikaty błędów}

\begin{itemize}
	\item W przypadku nie podania nazwy pliku z danymi wejściowymi:\\
	      "Lack of a configuration file.
	      Correct usage: path.game\_of\_life.exe config\_file"

	\item Podania nazwy pliku z danymi wejściowymi który 
	nie istnieje:\\
	      "No file 'FileName'."

	\item Braku potrzebnych danych w pliku:\\
	      "Configuration incomplete! Required values:\\
	      - height\\
	      - width\\
	      - png \textless{}instead of height and width\textgreater{}\\
	      - max\_iter\\
	      - freq or fps"
	
	\item Podania niepoprawnych koordynat żywych komórek:\\
	      "Y index too high! (Indexing starts at 0)"
	      
	\item Podania nazwy pliku png który nie istnieje:\\
		  "No file with name: FileName!"
		  
	\item Wystąpi błąd przy zapisie plików png:\\
		  "An error occured while writing file: FileName"

\end{itemize}
Jeżeli wystąpią powyższe błędy program kończy działanie.

\end{document}